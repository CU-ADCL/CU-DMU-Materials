\documentclass{article}
\usepackage{fullpage,amsmath,amsthm,graphicx,enumitem,amssymb,booktabs}
\usepackage[hidelinks]{hyperref}
\theoremstyle{definition}
\newtheorem{thm}{Theorem}
\newtheorem{question}[thm]{Question}
\newenvironment{solution}{\noindent\textit{Solution:}}{}

\newcommand{\reals}{\mathbb{R}}
\newcommand{\ttt}[1]{\texttt{#1}}

\title{ASEN 5519-003 Decision Making under Uncertainty\\
       Homework 5: Introduction to POMDPs and Advanced RL}

\begin{document}

\maketitle

\section{Exercises}

\begin{question}
    (25 pts) Consider the following POMDP that represents cancer monitoring and treatment plan\footnote{Note that the probabilities are not meant to be realistic. See \url{https://pubsonline.informs.org/doi/10.1287/opre.1110.1019} for an actual publication on this topic}:
\begin{eqnarray*}
    \mathcal{S} &=& \left\{\texttt{healthy}, \texttt{in-situ-cancer}, \texttt{invasive-cancer}, \texttt{death}\right\}\\
    \mathcal{A} &=& \left\{\texttt{wait}, \texttt{test}, \texttt{treat}\right\} \quad \mathcal{O} = \left\{\texttt{positive}, \texttt{negative}\right\}\\
    \gamma &=& 0.99 \quad
    s_0 = \texttt{healthy}
\end{eqnarray*}

The \textbf{transition dynamics} are designated with the following table. The state stays the same except with the probabilities encoded in the table.
\begin{center}
\begin{tabular}{ccc}
    \toprule
    $s$ & $a$ & $s'$: $\mathcal{T}(s' \mid s, a)$ \\
    \midrule
    \ttt{healthy} & all & \ttt{in-situ-cancer}: 2\% \\
    \ttt{in-situ-cancer} & \ttt{treat} & \ttt{healthy}: 60\% \\
    \ttt{in-situ-cancer} & $\neq\ttt{treat}$ & \ttt{invasive-cancer}: 10\% \\
    \ttt{invasive-cancer} & \ttt{treat} & \ttt{healthy}: 20\%; \ttt{death}: 20\% \\
    \ttt{invasive-cancer} & $\neq\ttt{treat}$ & \ttt{death}: 60\% \\
    \bottomrule
\end{tabular}
\end{center}

The \textbf{observation} is generated according to the following table. The observation is \ttt{negative} except with the probabilities encoded in the table.
\begin{center}
    \begin{tabular}{ccc}
    \toprule
    $a$ & $s'$ & $o$: $\mathcal{Z}(o \mid a, s')$ \\
    \midrule
    \texttt{test} & \texttt{healthy} & \texttt{positive}: 5\% \\
    \texttt{test} & \texttt{in-situ-cancer} & \texttt{positive}: 80\% \\
    \texttt{test} & \texttt{invasive-cancer} & \texttt{positive}: 100\% \\
    \texttt{treat} & \ttt{in-situ-cancer} or \texttt{invasive-cancer} & \texttt{positive}: 100\% \\
    \bottomrule
    \end{tabular}
\end{center}

The \textbf{rewards} are defined as follows (one could interpret the reward as roughly quality years of life):
\begin{itemize}[noitemsep]
    \item $R(\texttt{death}, \text{any action}) = 0.0$ (i.e. \texttt{death} is a terminal state)
    \item $R(\text{any living state}, \texttt{wait}) = 1.0$
    \item $R(\text{any living state}, \texttt{test}) = 0.8$ (because of costs and anxiety about a positive result)
    \item $R(\text{any living state}, \texttt{treat}) = 0.1$
\end{itemize}

Create a model of this problem using \texttt{QuickPOMDPs} and use Monte Carlo simulations to evaluate a policy that always \textbf{wait}s (we will solve this problem in the next homework).

\end{question}

\begin{question}
(25 pts) Using the deep learning library of your choice (e.g. Flux.jl, Knet.jl, Tensorflow, PyTorch), fit a neural network to approximate the function $f(x) = (1-x) \sin(20 \log(x+0.2))$ for the range $x \in [0,1]$. Plot a set of 100 data points fed through the trained model and plot the learning curve (loss vs number of training epochs).
\end{question}

\section{Challenge Problem}

\begin{question}
    (50 pts) In this exercise, you will learn a policy for the special mountain car environment, \texttt{DMUStudent.HW5.mc}. This environment is a standard mountain car with $\mathcal{A}=[-1, 1]$, except that the observation is a 100x100 matrix. The first two entries in the matrix contain the car's position and velocity. The rest of the matrix may contain useful information to receive additional rewards, but it is not needed to achieve full credit.
    \begin{enumerate}[label=\alph*)]
        \item Implement a reinforcement learning algorithm to learn a policy for the environment and plot a learning curve. Write a paragraph describing the algorithm you implemented.\footnote{I recommend discretizing the action space and implementing the DQN algorithm with only the position and velocity - see the starter code for instructions on how to create an environment wrapper to do this; this is the only algorithm that I can provide full debugging support for. DQN should be able to learn a policy that can achieve a return of in the 40s with a discount factor of $\gamma=0.99$ in less than 10 minutes of training time.}\label{q:impl}
        \item Evaluate a policy with \texttt{DMUStudent.HW5.evaluate}, and submit the resulting json file. You may use your code from part (\ref{q:impl} or \emph{any} other libraries for this part. A discount factor of $\gamma=0.99$ is used for evaluation. A score of 40 or greater will receive full credit. Your submission should be a function that takes in a state and returns an action.
    \end{enumerate}
\end{question}
\end{document}
