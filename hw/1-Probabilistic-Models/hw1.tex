\documentclass{article}
\usepackage{fullpage,amsmath,amsthm,graphicx,enumitem}
\usepackage{multicol}
\usepackage{booktabs}
\usepackage{hyperref}

\theoremstyle{definition}
\newtheorem{thm}{Theorem}
\newtheorem{question}[thm]{Question}
\newenvironment{solution}{\noindent\textit{Solution:}}{}

\title{ASEN 5519-003 Decision Making under Uncertainty\\
       Homework 1: Probabilistic Models}

\begin{document}

\maketitle

\section{Questions}

\begin{question} (20 pts)
    Consider the following joint distribution of three binary-valued random variables, \mbox{$A$, $B$, and $C$}:

    \begin{minipage}{0.3\linewidth}
        \vspace{1em}
    {\small
    \begin{tabular}{cccc}
        \toprule
            $A$ & $B$ & $C$ & $P(A,B,C)$ \\
        \midrule
            $0$ & $0$ & $1$ & $0.15$ \\
            $0$ & $1$ & $0$ & $0.05$ \\
            $0$ & $1$ & $1$ & $0.01$ \\
            $1$ & $0$ & $0$ & $0.14$ \\
            $1$ & $0$ & $1$ & $0.18$ \\
            $1$ & $1$ & $0$ & $0.29$ \\
            $1$ & $1$ & $1$ & $0.07$ \\
        \bottomrule
    \end{tabular}
    }
    \end{minipage}
    \begin{minipage}{0.7\linewidth}
        \begin{enumerate}[label=\alph*)]
            \item What is the probability of the outcome $A=0$, $B=0$, $C=0$?
            \item What is the marginal distribution of $A$?
            \item What is the conditional distribution of $A$ given $B=0$ and $C=1$?
        \end{enumerate}
    \end{minipage}
\end{question}

% \begin{question} (20 pts)
%     Let $B$ be a uniformly-distributed binary random variable and let $A$ be a real-valued random variable with the following conditional distribution:
%     $$A \mid B=0 \quad \sim \quad \mathcal{U}(0,2)$$
%     $$A \mid B=1 \quad \sim \quad \mathcal{U}(1,3)$$
%     \begin{enumerate}[nosep,label=(\alph*)]
%         \item Plot or draw the probability density functions for the conditional distribution of $A$.
%         \item Plot or draw the marginal density function of $A$.
%         \item What is the probability that $A=1.5$?
%         \item What is the probability that $A \in [1.5, 1.6]$?
%     \end{enumerate}
% \end{question}

\begin{question} (20 pts)
    2\% of women at age forty who participate in routine screening have breast cancer. 86\% of those with breast cancer will get positive mammograms. 8\% of those without breast cancer will also get positive mammograms. A woman in this age group had a positive mammogram in a routine screening. What is the probability that she actually has breast cancer?
\end{question}

\begin{question} (40 pts)
Suppose that a stationary stochastic process $\{x_t\}$ is defined by the following equation: $x_{t+1} = 1.5 \, x_t - x_{t-1} + v_{t}$ where $v_t$ are independent, identically distributed random variables with $v_t \sim \mathcal{N}(0.0, 0.2^2)$.
    \begin{enumerate}[nosep,label=(\alph*)]
        \item Simulate and plot 10 20-step trajectories sampled from this process with $x_0 = x_{-1} = 1$ (You may use any programming language, but as always, submit your code).
        \item Is this process a Markov process if the state is defined as $x_t$? Why or why not?
        \item If you only had access to the trajectories you plotted what evidence could you use to convince someone that this process is or is not Markov?
        \item What would need to be included in the state at time $t$ to make this a Markov process?
    \end{enumerate}
\end{question}

(continues on next page)

\pagebreak

\section{Auto-graded Programming}

\begin{question} (20 pts)
    In this exercise, you will write and test a Julia function to ensure that you can get Julia and the course-specific code running and help you learn how to do a task that sometimes trips students up in homework 2. Your function should take two arguments:
    \begin{itemize}[nosep]
        \item \texttt{a}: a matrix, and
        \item \texttt{bs}: a non-empty vector of vectors.
    \end{itemize}
    The function should multiply all of the vectors in \texttt{bs} by \texttt{a} and then return the \emph{elementwise} maximum of the resulting vectors.

    In order to get full-credit, the function must be completely ``type-stable'' (see the \href{https://docs.julialang.org/en/v1/manual/performance-tips/#Write-%22type-stable%22-functions}{``Performance Tips'' section of the Julia manual}). Your function should always return a vector with the same element type as \texttt{a}. You can assume the vectors in \texttt{bs} will have the same element type as \texttt{a}, but you should be able to handle \texttt{a} with any numeric element type.

    Evaluate this function with \texttt{DMUStudent.HW1.evaluate} and submit the resulting json file \textit{along with a listing of the code}. A score of 1 will receive full credit.
\end{question}

\end{document}
