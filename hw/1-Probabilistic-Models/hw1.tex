\documentclass{article}
\usepackage{fullpage,amsmath,amsthm,graphicx,enumitem}
\usepackage{multicol}
\usepackage{booktabs}

\theoremstyle{definition}
\newtheorem{thm}{Theorem}
\newtheorem{question}[thm]{Question}
\newenvironment{solution}{\noindent\textit{Solution:}}{}

\title{ASEN 5519-003 Decision Making under Uncertainty\\
       Homework 1: Probabilistic Models}

\begin{document}

\maketitle

\section{Conceptual Questions}

\begin{question} (20 pts)
    Consider the following joint distribution of three binary-valued random variables, \mbox{$A$, $B$, and $C$}:

    \begin{minipage}{0.23\linewidth}
        \vspace{1em}
    {\tiny
    \begin{tabular}{cccc}
        \toprule
            $A$ & $B$ & $C$ & $P(A,B,C)$ \\
        \midrule
            $0$ & $0$ & $0$ & $0.08$ \\
            $0$ & $0$ & $1$ & $0.15$ \\
            $0$ & $1$ & $0$ & $0.05$ \\
            $0$ & $1$ & $1$ & $0.10$ \\
            $1$ & $0$ & $0$ & $0.14$ \\
            $1$ & $0$ & $1$ & $0.18$ \\
            $1$ & $1$ & $0$ & $0.19$ \\
            $1$ & $1$ & $1$ & $0.11$ \\
        \bottomrule
    \end{tabular}
    }
    \end{minipage}
    \begin{minipage}{0.75\linewidth}
        \begin{enumerate}[label=\alph*)]
            \item What is the marginal distribution of $A$?
            \item What is the conditional distribution of $A$ given $B=1$ and $C=1$?
        \end{enumerate}
    \end{minipage}
\end{question}

\begin{question} (20 pts)
    1\% of women at age forty who participate in routine screening have breast cancer. 80\% of those with breast cancer will get positive mammographies. 9.6\% of those without breast cancer will also get positive mammographies. A woman in this age group had a positive mammography in a routine screening. What is the probability that she actually has breast cancer?
\end{question}

\begin{question} (10 pts)
    Suppose that a stochastic process $\{x_t\}$ is defined by the following equation: $x_{t+1} = x_t + x_{t-1} + v_{t}$ where $v_t$ are i.i.d. noise random variables.
    \begin{enumerate}[label=(\alph*)]
        \item Is this process Markov if the state is defined as $x_t$?
        \item What would need to be included in the state at time $t$ to make this a Markov process?
    \end{enumerate}
\end{question}

\section{Exercises}

\begin{question} (30 pts)
    Consider two stochastic processes, $\{x_t\}$ and $\{y_t\}$, defined by $x_t = f_x(x_{0}, x_{1}... , x_{t-1}, v_t)$ and $y_t = f_y(y_{0}, y_{1}... , y_{t-1}, v_t)$ where $v_t$ are independent, identically distributed random variables that introduce noise. The \texttt{HW1} module in \texttt{DMUStudent} contains two Julia functions, \texttt{fx} and \texttt{fy}, which can sample from the stochastic processes, i.e. \texttt{fx([x1, x2])} will return a sample of $x_3$ given that $x_1 = \texttt{x1}$ and $x_2 = \texttt{x2}$.

    \begin{enumerate}[label=\alph*)]
        \item Plot a trajectory of $\{x\}$ and $\{y\}$.
        \item Suppose you know that one of these processes is Markov and the other is not. By drawing samples with the Julia functions, determine which one is Markov.
    \end{enumerate}
\end{question}

\section{Challenge Problem}

\begin{question} (20 pts)
    Write a function in Julia that takes in two arguments and uses the Pythagorean Theorem to compute the hypotenuse of a right triangle with side lengths specified by the arguments. Evaluate this function with \texttt{DMUStudent.HW1.evaluate} and submit the resulting json file \textit{along with a listing of the code in your pdf}. A score of 1 will receive full credit.\footnote{This particular ``Challenge Problem'' is not meant to be challenging; it is meant to test that everyone can download the code and submit. It should require only 2 lines. Future problems in this section will be quite challenging.}
\end{question}

\end{document}
