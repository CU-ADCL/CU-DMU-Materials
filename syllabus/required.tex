\subsection*{Classroom Behavior}

Students and faculty are responsible for maintaining an appropriate
learning environment in all instructional settings, whether in person,
remote, or online. Failure to adhere to such behavioral standards may be
subject to discipline. Professional courtesy and sensitivity are
especially important with respect to individuals and topics dealing with
race, color, national origin, sex, pregnancy, age, disability, creed,
religion, sexual orientation, gender identity, gender expression,
veteran status, political affiliation, or political philosophy.\\
~\\
For more information, see the
\href{http://www.colorado.edu/policies/student-classroom-and-course-related-behavior}{{classroom
behavior policy}}, the
\href{https://www.colorado.edu/sccr/student-conduct}{{Student Code of
Conduct}}, and the \href{https://www.colorado.edu/oiec/}{{Office of
Institutional Equity and Compliance}}.

\subsection*{Requirements for Infectious Disease}

Members of the CU Boulder community and visitors to campus must follow
university, department, and building health and safety requirements and
all applicable campus policies and public health guidelines to reduce
the risk of spreading infectious diseases. If public health conditions
require, the university may also invoke related requirements for student
conduct and disability accommodation that will apply to this class.

If you feel ill and think you might have COVID-19 or if you have tested
positive for COVID-19, please stay home and follow the
\href{https://www.cdc.gov/coronavirus/2019-ncov/your-health/isolation.html}{{guidance
of the Centers for Disease Control and Prevention (CDC) for isolation
and testing}}. If you have been in close contact with someone who has
COVID-19 but do not have any symptoms and have not tested positive for
COVID-19, you do not need to stay home but should follow the
\href{https://www.cdc.gov/coronavirus/2019-ncov/your-health/if-you-were-exposed.html}{{guidance
of the CDC for masking and testing}}.

\subsection*{Accommodation for Disabilities, Temporary Medical Conditions, and
Medical Isolation}

If you qualify for accommodations because of a disability, please submit
your accommodation letter from Disability Services to your faculty
member in a timely manner so that your needs can be addressed.~
Disability Services determines accommodations based on documented
disabilities in the academic environment.~ Information on requesting
accommodations is located on
the~\href{https://www.colorado.edu/disabilityservices/}{{Disability
Services website}}. Contact Disability Services at 303-492-8671
or~\href{mailto:dsinfo@colorado.edu}{{dsinfo@colorado.edu}}~ for further
assistance.~ If you have a temporary medical condition,
see~\href{https://www.colorado.edu/disabilityservices/students/temporary-medical-conditions}{{Temporary
Medical Conditions}}~on the Disability Services website.

Students are expected to start on assignments early so that minor temporary medical conditions do not prevent them from turning assignments in on time. In addition, the late policy is designed to accommodate minor temporary medical conditions. If you have a major medical emergency that prevents you from completing an assignment, please contact the instructor as soon as possible to discuss accommodations.

\subsection*{Preferred Student Names and Pronouns}

CU Boulder recognizes that students' legal information doesn't always
align with how they identify. Students may update their preferred names
and pronouns via the student portal; those preferred names and pronouns
are listed on instructors' class rosters. In the absence of such
updates, the name that appears on the class roster is the student's
legal name.

\subsection*{Honor Code}

All students enrolled in a University of Colorado Boulder course are
responsible for knowing and adhering to the
\href{https://www.colorado.edu/sccr/honor-code}{{Honor Code}}.
Violations of the Honor Code may include but are not limited to:
plagiarism (including use of paper writing services or technology
{[}such as essay bots{]}), cheating, fabrication, lying, bribery,
threat, unauthorized access to academic materials, clicker fraud,
submitting the same or similar work in more than one course without
permission from all course instructors involved, and aiding academic
dishonesty.

All incidents of academic misconduct will be reported to Student Conduct
\& Conflict Resolution:
\href{mailto:honor@colorado.edu}{{honor@colorado.edu}}, 303-492-5550.
Students found responsible for violating the
\href{https://www.colorado.edu/sccr/honor-code}{{Honor Code}} will be
assigned resolution outcomes from the Student Conduct \& Conflict
Resolution as well as be subject to academic sanctions from the faculty
member. Visit \href{https://www.colorado.edu/sccr/honor-code}{{Honor
Code}} for more information on the academic integrity policy.

\subsection*{Sexual Misconduct, Discrimination, Harassment and/or Related Retaliation}

CU Boulder is committed to fostering an inclusive and welcoming
learning, working, and living environment. University policy prohibits
\href{https://www.colorado.edu/oiec/policies/discrimination-harassment-policy/protected-class-definitions}{{protected-class}}
discrimination and harassment, sexual misconduct (harassment,
exploitation, and assault), intimate partner violence (dating or
domestic violence), stalking, and related retaliation by or against
members of our community on- and off-campus. These behaviors harm
individuals and our community. The Office of Institutional Equity and
Compliance (OIEC) addresses these concerns, and individuals who have
been subjected to misconduct can contact OIEC at 303-492-2127 or email
\href{mailto:cureport@colorado.edu}{{cureport@colorado.edu}}.
Information about university policies,
\href{https://www.colorado.edu/oiec/reporting-resolutions/making-report}{{reporting
options}}, and
\href{https://www.colorado.edu/oiec/support-resources}{{support
resources}} can be found on the
\href{http://www.colorado.edu/institutionalequity/}{{OIEC website}}.

Please know that faculty and graduate instructors must inform OIEC when
they are made aware of incidents related to these policies regardless of
when or where something occurred. This is to ensure that individuals
impacted receive outreach from OIEC about resolution options and support
resources. To learn more about reporting and support for a variety of
concerns, visit the {\href{https://www.colorado.edu/dontignoreit/}{
Don't Ignore It} page}.

\subsection*{Religious Accommodations}

Campus policy requires faculty to provide reasonable accommodations for
students who, because of religious obligations, have conflicts with
scheduled exams, assignments or required attendance. Please communicate
the need for a religious accommodation in a timely manner, i.e. at least a week before the event. 

See the
\href{http://www.colorado.edu/policies/observance-religious-holidays-and-absences-classes-andor-exams}{{campus
policy regarding religious observances}} for full details.

\subsection*{Mental Health and Wellness}

The University of Colorado Boulder is committed to the well-being of all
students. If you~are struggling with personal stressors, mental health
or substance use concerns that are impacting academic or daily life,
please contact~\href{https://www.colorado.edu/counseling/}{{Counseling
and Psychiatric Services (CAPS)}} located in C4C or call (303) 492-2277,
24/7.~\\
~\\
Free and unlimited telehealth is also available
through~\href{https://www.colorado.edu/health/academiclivecare}{{Academic
Live Care}}.
The~\href{https://www.colorado.edu/health/academiclivecare}{Academic
Live Care} site also provides information about additional wellness
services on campus that are available to students.
