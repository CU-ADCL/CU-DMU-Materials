\subsection*{Honor Code}
All students enrolled in a University of Colorado Boulder course are responsible for knowing and adhering to the \href{https://www.colorado.edu/sccr/media/229}{Honor Code}. Violations of the Honor Code may include but are not limited to: plagiarism (including use of paper writing services or technology [such as essay bots]), cheating, fabrication, lying, bribery, threat, unauthorized access to academic materials, clicker fraud, submitting the same or similar work in more than one course without permission from all course instructors involved, and aiding academic dishonesty. Understanding the course's syllabus is a vital part in adhering to the Honor Code.

All incidents of academic misconduct will be reported to Student Conduct \& Conflict Resolution: \href{mailto:StudentConduct@colorado.edu}{StudentConduct@colorado.edu}. Students found responsible for violating the \href{https://www.colorado.edu/sccr/media/229}{Honor Code} will be assigned resolution outcomes from the Student Conduct \& Conflict Resolution as well as be subject to academic sanctions from the faculty member. Visit \href{https://www.colorado.edu/sccr/media/229}{Honor Code} for more information on the academic integrity policy. 

\subsection*{Accommodation for Disabilities, Temporary Medical Conditions, and Medical Isolation:}
If you qualify for accommodations because of a disability, please submit your accommodation letter from Disability Services to your faculty member in a timely manner so that your needs can be addressed. \textbf{Students should expect to receive accommodations for a timed assessment (e.g., exam) only if their faculty instructor(s) receive the student's accommodations letter at least 5 business days before the assessment, as a departmental policy, in order to facilitate administering the assessment.} Disability Services determines accommodations based on documented disabilities in the academic environment.  Information on requesting accommodations is located on the \href{https://www.colorado.edu/disabilityservices/}{Disability Services website}. Contact Disability Services at 303-492-8671 or \href{mailto:DSinfo@colorado.edu}{DSinfo@colorado.edu} for further assistance. If you have a temporary medical condition, see \href{https://www.colorado.edu/disabilityservices/students/temporary-medical-conditions}{Temporary Medical Conditions} on the Disability Services website.

If you have a temporary medical condition or required medical isolation for which you require accommodation, please notify the instructor as soon as possible so that appropriate accommodations can be made.

If you are sick or require isolation please notify the instructor of your absence from in-person activities and continue in a completely remote mode, as you are able, until you are allowed or able to return to campus.

Please note that for health privacy reasons you are not required to disclose to the instructor the nature of your illness or condition, however you are welcome to share information you feel necessary to protect the health and safety of others within the course.

\subsection*{Accommodation for Religious Obligations:}
Campus policy requires faculty to provide reasonable accommodations for students who, because of religious obligations, have conflicts with scheduled exams, assignments or required attendance. Please communicate the need for a religious accommodation in a timely manner. In this class, you must let the instructor know of any such conflicts within the first two weeks of the semester so that they can work with you to make reasonable arrangements.
See the \href{https://www.colorado.edu/compliance/policies/observance-religious-holidays-absences-classes-or-exams}{campus policy regarding religious observances} for full details.

\subsection*{Preferred Student Names and Pronouns:}
CU Boulder recognizes that students' legal information doesn't always align with how they identify. Students may update their preferred names and pronouns via the student portal; those preferred names and pronouns are listed on instructors' class rosters. In the absence of such updates, the name that appears on the class roster is the student's legal name.

\subsection*{Classroom Behavior:}
Students and faculty are responsible for maintaining an appropriate learning environment in all instructional settings, whether in person, remote, or online. Failure to adhere to such behavioral standards may be subject to discipline. Professional courtesy and sensitivity are especially important with respect to individuals and topics dealing with race, color, national origin, sex, pregnancy, age, disability, creed, religion, sexual orientation, gender identity, gender expression, veteran status, marital status, political affiliation, or political philosophy.
  
For more information, see the \href{https://www.colorado.edu/compliance/policies/student-classroom-course-related-behavior}{classroom behavior policy}, the \href{https://www.colorado.edu/sccr/media/230}{Student Code of Conduct}, and the \href{https://www.colorado.edu/oiec/}{Office of Institutional Equity and Compliance}. 

\subsection*{Sexual Misconduct, Discrimination, Harassment and/or Related Retaliation:}
CU Boulder is committed to fostering an inclusive and welcoming learning, working, and living environment. University policy prohibits \href{https://www.colorado.edu/oiec/policies/discrimination-harassment-policy/protected-class-definitions}{protected-class} discrimination and harassment, sexual misconduct (harassment, exploitation, and assault), intimate partner abuse (dating or domestic violence), stalking, and related retaliation by or against members of our community on- and off-campus. The Office of Institutional Equity and Compliance (OIEC) addresses these concerns, and individuals who have been subjected to misconduct can contact OIEC at 303-492-2127 or email \href{mailto:CUreport@colorado.edu}{CUreport@colorado.edu}. Information about university policies, \href{https://www.colorado.edu/oiec/reporting-resolutions/making-report}{reporting options}, and \href{OIEC support resources} including confidential services can be found on the \href{http://www.colorado.edu/institutionalequity/}{OIEC website}.

Please know that faculty and graduate instructors are required to inform OIEC when they are made aware of incidents related to these concerns regardless of when or where something occurred. This is to ensure that individuals impacted receive outreach from OIEC about their options and support resources. To learn more about reporting and support for a variety of concerns, visit the \href{https://www.colorado.edu/dontignoreit/}{Don't Ignore It page}.


\subsection*{Mental Health and Wellness:}
The University of Colorado Boulder is committed to the well-being of all students. If you are struggling with personal stressors, mental health or substance use concerns that are impacting academic or daily life, please contact \href{https://www.colorado.edu/counseling/}{Counseling and Psychiatric Services (CAPS)} located in C4C or call (303) 492-2277, 24/7.

Free and unlimited telehealth is also available through \href{https://www.colorado.edu/health/academiclivecare}{Academic Live Care}. The Academic Live Care site also provides information about additional wellness services on campus that are available to students.
